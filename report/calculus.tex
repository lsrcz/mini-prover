\section{Calculus}

\subsection{Term}

\begin{enumerate}
    \item $\tt Set,Prop$ are terms.
    \item Variables $\tt x,y$, etc., are terms.
    \item Constants $\tt c,d$, etc., are terms.
    \item If $\tt x$ is a variable and $\tt T,U$ are terms, then $\forall\tt x:T,U$ is a term.
    \item If $\tt x$ is a variable and $\tt T,u$ are terms, then $\lambda\tt x:T.\ u$ is a term.
    \item If $\tt x$ and $\tt u$ are terms, then $\tt (t\ u)$ is a term.
    \item If $\tt x$ is a variable and $\tt t,T,u$ are terms, then $\tt let\ \ x:=t:T\ \ in\ \ u$ is a term.
\end{enumerate}

\subsection{Typing Rule}

\subsubsection{Notation}
\begin{itemize}
\item $\mathcal S$ : $\{\tt Prop,Set\}$.
\item $E$ : global environment.
\item $\Gamma$ : local context.
\item $\tt u\{x/t\}$ : substitute free occurrence of variable $\tt x$ to term $\tt t$ in term $\tt u$.
\item $\mathcal{WF}(E)[\Gamma]$ : $E$ is well-formed and $\Gamma$ is valid in $E$.
\end{itemize}

\subsubsection{Typing Rules}
\begin{equation*}
\mathcal{WF}([])[] 
    \tag{\sc T-Empty}
\end{equation*}
\begin{equation*}
\frac{E[\Gamma]\vdash {\tt T:s\qquad s\in\mathcal{S}\qquad x}\notin\Gamma}
    {\mathcal{WF}(E)[\Gamma ::(\tt x:T)]} 
    \tag{\sc T-Local-Ax}
\end{equation*}
\begin{equation*}
\frac{E[]\vdash {\tt t:T\qquad c}\notin E}
    {\mathcal{WF}(E:\tt c:=t:T)} 
    \tag{\sc T-Local-Def}
\end{equation*}
\begin{equation*}
\frac{\mathcal{WF}(E)[\Gamma]\qquad ({\tt x:T})\in\Gamma}
    {E[\Gamma]\vdash\tt x:T} 
    \tag{\sc T-Var1}
\end{equation*}
\begin{equation*}
\frac{\mathcal{WF}(E)[\Gamma]\qquad ({\tt x:=t:T})\in\Gamma}
    {E[\Gamma]\vdash\tt x:T} 
    \tag{\sc T-Var2}
\end{equation*}
\begin{equation*}
\frac{\mathcal{WF}(E)[\Gamma]\qquad ({\tt c:T})\in E}
    {E[\Gamma]\vdash\tt c:T} 
    \tag{\sc T-Const1}
\end{equation*}
\begin{equation*}
\frac{\mathcal{WF}(E)[\Gamma]\qquad ({\tt c:=t:T})\in E}
    {E[\Gamma]\vdash\tt c:T} 
    \tag{\sc T-Const2}
\end{equation*}

\begin{equation*}
\frac{E[\Gamma]\vdash{\tt T:s\qquad s\in\mathcal{S}}\qquad E[\Gamma::({\tt x:T})]\vdash\tt U:Prop}
    {E[\Gamma]\vdash\tt\forall x:T,U:Prop} 
    \tag{\sc T-Prod-Prop}
\end{equation*}
\begin{equation*}
\frac{E[\Gamma]\vdash{\tt T:s\qquad s\in\mathcal{S}}\qquad E[\Gamma::({\tt x:T})]\vdash\tt U:Set}
    {E[\Gamma]\vdash\tt\forall x:T,U:Set} 
    \tag{\sc T-Prod-Set}
\end{equation*}
\begin{equation*}
\frac{E[\Gamma]\vdash\forall{\tt x:T,U:s}\qquad E[\Gamma::({\tt x:T})]\vdash{\tt t:U}}
    {E[\Gamma]\vdash\lambda{\tt x:T.\ t:\forall x:T,U}} 
    \tag{\sc T-Abs}
\end{equation*}
\begin{equation*}
\frac{E[\Gamma]\vdash\forall{\tt x:U,T}\qquad E[\Gamma]\vdash{\tt u:U}}
    {E[\Gamma]\vdash{\tt (t\ u):T\{x/u\}}} 
    \tag{\sc T-App}
\end{equation*}
\begin{equation*}
\frac{E[\Gamma]\vdash{\tt t:T}\qquad E[\Gamma::({\tt x:=t:T})]\vdash{\tt u:U}}
    {E[\Gamma]\vdash{\tt let\ \ x:=t:T\ \ in\ \ u:U\{x/t\}}} 
    \tag{\sc T-Let}
\end{equation*}

\subsection{Conversion Rule}

\subsubsection{Notation}
\begin{itemize}
    \item $E[\Gamma]\vdash \tt t\rhd u$ : $\tt t$ reduces to $\tt u$ in $E,\Gamma$ with 
    one of the $\beta,\iota,\delta,\zeta$ reductions.
    \item $E[\Gamma]\vdash \tt t\mathop{\rhd}\limits^* u$ : $E[\Gamma]\vdash\tt t\rhd\cdots\rhd u$.
    \item $\tt u\equiv v$ : $\tt u$ and $\tt v$ are identical.
\end{itemize}

\subsubsection{Conversion Rules}
\begin{equation*}
\frac{E[\Gamma]\vdash(\lambda{\tt x:T.\ t})\ u}
    {\tt t\{x/u\}}
    \tag{\sc $\beta$-Conv}
\end{equation*}
\begin{equation*}
\frac{\tt case((c_p\ q_1\ \cdots\ q_r\ a_1\ \cdots\ a_m),f_1|\cdots|f_n)}
    {f_i\ a_1\ \cdots\ a_m}
    \tag{\sc $\iota$-Conv}
\end{equation*}
\begin{equation*}
\frac{E[\Gamma]\vdash{\tt x}\qquad ({\tt x:=t:T})\in\Gamma}
    {\tt t}
    \tag{\sc $\delta$-Conv1}
\end{equation*}
\begin{equation*}
\frac{E[\Gamma]\vdash{\tt c}\qquad ({\tt x:=t:T})\in E}
    {\tt t}
    \tag{\sc $\delta$-Conv2}
\end{equation*}
\begin{equation*}
\frac{E[\Gamma]\vdash \tt let\ \ x:=u\ \ in\ \ t}
    {\tt t\{x/u\}}
    \tag{\sc $\zeta$-Conv}
\end{equation*}
\begin{equation*}
\frac{E[\Gamma]\vdash \tt t:\forall x:T,U\qquad x\text{ fresh in }t}
    {\tt \lambda\tt x:T.\ (t\ x)}
    \tag{\sc $\eta$-Exp}
\end{equation*}

\begin{Def}[Convertibility]
$\tt t_1$ and $\tt t_2$ are convertible iff there exists $\tt u_1$ and $\tt u_2$ such that 
$E[\Gamma]\vdash \tt t_1\mathop{\rhd}\limits^* u_1$ and $E[\Gamma]\vdash \tt t_2\mathop{\rhd}\limits^* u_2$ 
and either $u_1\equiv u_2$ or they are convertible up to $\eta$-expansion.
\end{Def}

\subsection{Inductive Definition}

\subsubsection{Notation}
\begin{itemize}
    \item ${\tt Ind}[p](\Gamma_I:=\Gamma_C)$ : inductive definition.
    \item $\Gamma_I$ : names and types of inductive type.
    \item $\Gamma_C$ : names and types of constructors of inductive type.
    \item $p$ : the number of parameters of inductive type.
    \item $\Gamma_P$ : the context of parameters.
\end{itemize}

\subsubsection{Typing Rule}
\begin{equation*}
\frac{\mathcal{WF}(E)[\Gamma]\qquad {\tt Ind}[p](\Gamma_I:=\Gamma_C)\in E\qquad {\tt(a:A)}\in\Gamma_i}
    {E[\Gamma]\vdash\tt a:A}
    \tag{\sc T-Ind}
\end{equation*}
\begin{equation*}
\frac{\mathcal{WF}(E)[\Gamma]\qquad {\tt Ind}[p](\Gamma_I:=\Gamma_C)\in E\qquad {\tt (c:C)}\in\Gamma_C}
    {E[\Gamma]\vdash\tt c:C}
    \tag{\sc T-Constr}
\end{equation*}
\begin{equation*}
\frac{(E[\Gamma_P]\vdash{\tt A_j:s_j'})_{j=1..k}\qquad (E[\Gamma_i;\Gamma_P]\vdash{\tt C_i:s_{q_i}})_{i=1..n}}
    {\mathcal{WF}(E;{\tt Ind}[p](\Gamma_I:=\Gamma_C))[\Gamma]}
    \tag{\sc T-Wf-Ind}
\end{equation*}

\subsubsection{Well-formed Requirement}

To maintain the consistency of the system, we must restrict the inductive definitions to a syntactic criterion of 
\textbf{positivity}, which guarantees the \textit{soundness and safety} of the system.

\begin{Def}[Constructor] 
$\tt T$ is a \textit{type of constructor of }$\tt I$ if
    \begin{itemize}\normalfont
        \item $\tt T\equiv(\tt I\ t_1\ \cdots\ t_n)$
        \item $\tt T\equiv\forall x:U,T'$, where $T'$ is a type of constructor of $\tt I$ 
    \end{itemize}
\end{Def}

\begin{Def}[Positivity]
The type of constructor $\tt T$ satisfies \textit{the positivity condition} for a constant $\tt X$ if
    \begin{itemize}\normalfont
        \item $\tt T\equiv (X\ t_1\ \cdots\ t_n)$ and $\tt X$ does not occur free in $\tt t_i$
        \item $\tt T\equiv \forall x:U,V$ and $\tt X$ occurs only \textit{strictly positively} in $\tt U$ and 
            $\tt V$ satisfies \textit{the positivity condition} for $\tt X$
    \end{itemize}
\end{Def}

\begin{Def}[Strictly Positivity]
The constant $\tt X$ occurs \textit{strictly positively} in $\tt T$ if 
    \begin{itemize}\normalfont
        \item $\tt X$ does not occur in $\tt T$
        \item $\tt T\mathop{\rhd}^*\tt (X\ t_1\ \cdots\ t_n)$ and $\tt X$ does not occur in $\tt t_i$
        \item $\tt T\mathop{\rhd}^*\tt \forall x:U,V$ and $\tt X$ does not occur in $\tt U$ but occurs
            \textit{strictly positively} in $\tt V$
        \item $\tt T\mathop{\rhd}^*\tt (I\ a_1\ \cdots\ a_m\ t_1\ \cdots\ t_p)$, where 
            ${\tt Ind}[m](\tt I:A:=c_1:\forall p_1:P_1,\dots\forall p_m:P_m,C_1;\cdots;c_n:\forall p_1:P_1,\dots,
            \forall p_m:P_m,C_n)$, and $\tt X$ does not occur in $t_i$, and the types of constructor 
            $\tt C_i\{p_j/a_j\}_{j=1..m}$ satisfies \textit{the nested positivity condition} for $\tt X$ 
    \end{itemize}
\end{Def}

\begin{Def}[Nested Positivity]
The type of constructor $\tt T$ satisfies \textit{the nested positivity condition} for a constant $\tt X$ if
    \begin{itemize}\normalfont
        \item $\tt T\equiv(I\ b_1\ \cdots\ b_m\ u_1\ \cdots\ u_p)$, where $\tt I$ is an inductive definition with $m$
            parameters and $\tt X$ does not occur in $u_i$
        \item $\tt T\equiv\forall x:U,V$ and $\tt X$ occurs \textit{strictly positively} in $\tt U$ and $\tt V$
            satisfies \textit{the nested positivity condition} for $\tt X$
    \end{itemize}
\end{Def}

\section{Sugar and Desugar}

\subsection{Match}
\begin{Def}[Desugar of \tt match]
    \begin{equation*}
    \frac{\tt match\ m\ with\ (c_1\ x_{11}\ \cdots\ x_{1p_1})\Rightarrow f_1\ |\ \cdots\ |\ (c_n\ x_{n1}\ \cdots\ x_{np_n})
        \Rightarrow f_n\ end}
        {\tt case(m,\lambda x_{11}\cdots x_{1p_1}.\ f_1\ |\ \cdots\ |\lambda x_{n1}\cdots x_{np_n}.\ f_n)}
        \tag{\sc D-Match}
    \end{equation*}
\end{Def}

TYPING RULE REQUIRES

\begin{equation*}
\frac{\tt case((c_p\ q_1\ \cdots\ q_r\ a_1\ \cdots\ a_m),f_1|\cdots|f_n)}
    {f_i\ a_1\ \cdots\ a_m}
    \tag{\sc $\iota$-Conv}
\end{equation*}

\subsection{Fixpoint}
\begin{Def}[Desugar of \tt fix]
    \begin{equation*}
    \frac{\tt fix\ f_1(\Gamma_1):A_1:=t_1\ with\cdots with\ f_n(\Gamma_n):A_n:=t_n\ for f_i}
        {qqq}
        \tag{\sc D-Fix}
    \end{equation*}
\end{Def}

\subsection{Let}

\begin{Def}[Desugar of \tt let]
\begin{equation*}
\frac{E[\Gamma]\vdash \tt let\ \ x:=u\ \ in\ \ t}
    {\tt t\{x/u\}}
    \tag{\sc $\zeta$-Conv}
\end{equation*}
\end{Def}
