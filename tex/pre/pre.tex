% !TeX encoding = UTF-8
% !TeX spellcheck = LaTeX
% !TeX program = XeLaTeX
%

% Please compile by 'xelatex -shell-escape pre.tex' 

\documentclass[UTF-8]{beamer}

\usepackage{graphicx}
\usepackage{tikz}
\usepackage{amsmath,amssymb}
\usepackage{bm}
\usepackage{extarrows}
\usepackage{newtxtext}
\usepackage{minted}
\undef\checkmark
\usepackage{dingbat}
\usepackage{listings}
\usepackage{caption}
\usepackage{xcolor}

\usetheme{CambridgeUS}
\usecolortheme{crane}
\usefonttheme{professionalfonts}
\setbeamertemplate{footjoline}[frame number]
\useoutertheme{infolines}

\definecolor{bg}{rgb}{0.95,0.95,0.95}
\renewcommand{\listingscaption}[1]{\begin{center}CODE: #1\end{center}\vspace{-20pt}}
\setminted{
    fontsize=\small, 
    bgcolor=bg, frame=lines,
    mathescape, breaklines, breakanywhere,
    breaksymbolleft=\raisebox{0.8ex}{\small\reflectbox{\carriagereturn}},
    breaksymbolright=\small\carriagereturn
}

\usetikzlibrary{shapes.geometric, arrows}
\tikzstyle{startstop} = [rectangle, rounded corners, minimum width=2.5cm, minimum height=1cm, text centered, 
draw=black, fill=red!30, font=\bf]
\tikzstyle{io} = [trapezium, trapezium left angle=70, trapezium right angle=110, minimum width=2.5cm, minimum height=1cm,
text centered, draw=black, fill=blue!30, font=\bf]
\tikzstyle{process} = [rectangle, minimum width=2.5cm, minimum height=1cm, text centered, 
draw=black, fill=orange!30, font=\bf]
\tikzstyle{decision} = [diamond, minimum width=2.5cm, minimum height=1cm, text centered, 
draw=black, fill=green!30, font=\bf]
\tikzstyle{arrow} = [thick, ->, >= stealth]
\tikzstyle{line} = [thick]

\title{MiniProver}
\subtitle{A coq-like proof assistant}
\institute{Peking University}
\author{Zhenwen Li, Sirui Lu, Kewen Wu}
\date{\today}

\begin{document}
\frame{\titlepage}

\section{Syntax}
\begin{frame}
\end{frame}

\section{Type System and Inference Rule}
\begin{frame}
\end{frame}

\section{Implementation}
\subsection{Termination}
\begin{frame}[fragile]{Termination Check}
\begin{minted}[escapeinside=<>]{coq}
Fixpoint plus (<\colorbox{green}{n}> m : nat) : nat :=
    match <\colorbox{green}{n}> with
    | O => m
    | S p => S (plus <\colorbox{green}{p}> m)

Fixpoint plus' (<\colorbox{green}{n}> <\colorbox{red}{m}> : nat) : nat :=
    match <\colorbox{green}{n}> with
    | O => m
    | S p => S (plus <\colorbox{red}{m}> <\colorbox{green}{p}>)
\end{minted}
\begin{alertblock}{Criterion}
    Descending on at least one variable.
\end{alertblock}
\end{frame}

\subsection{Inductive Positivity}
\begin{frame}[fragile]{Positivity Check}
\begin{minted}[escapeinside=<>]{coq}
Inductive ill : Type :=
    | malf : (ill -> ill) -> ill 

Definition extract (t : ill) : ill :=
    match t with
    | malf f => f t

    extract (malf extract)  (* not terminating *)
\end{minted}
\begin{alertblock}{Criterion}
    Types of constructors satisfy \textbf{positivity condition} for name of inductive type.
\end{alertblock}
\end{frame}

\subsection{Inductive Type and Term}
\begin{frame}[fragile]{Build Term and Type for Inductive Definition}
\begin{exampleblock}{Intuition}
    View mathematical induction as building a term for inductive definition.
\end{exampleblock}

\begin{minted}[escapeinside=<>]{coq}
Inductive nat : Type := O : nat | S : nat -> nat

fun (P : nat -> Type) (f : P O) 
        (f0 : forall n : nat, P n -> P (S n)) 
    fix F (n : nat) : P n :=
        match n as n0 in (nat) return (P n0) with
        | O => f
        | S n0 => f0 n0 (F n0)
:
forall P : nat -> Type, P O -> 
    (forall n : nat, P n -> P (S n)) -> 
        forall n: nat, P n
\end{minted}
\end{frame}

\section{Key Tactic}
\begin{frame}{Intro and Intros}
\end{frame}

\begin{frame}{Unfold}
\end{frame}

\begin{frame}{Rewrite}
\end{frame}

\begin{frame}{Apply}
\end{frame}

\begin{frame}{Destruct}
\end{frame}

\begin{frame}{Induction}
\end{frame}

\section{Demo}
\begin{frame}
\end{frame}


\section*{Thanks}
\begin{frame}
\LARGE
\begin{beamercolorbox}[center,ht=3em]{titlelike}
\vspace{1em}
Thanks!
\end{beamercolorbox}
\end{frame}\end{document}
