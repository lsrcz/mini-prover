\section{Typing}

\subsection{Term}

\begin{enumerate}
    \item $\tt Type$ is a term.
    \item Variables $\tt x,y$, etc., are terms.
    \item Constants $\tt c,d$, etc., are terms.
    \item If $\tt x$ is a variable and $\tt T,U$ are terms, then $\tt\forall x:T,U$ is a term.
    \item If $\tt x$ is a variable and $\tt T,u$ are terms, then $\tt\lambda x:T.\ u$ is a term.
    \item If $\tt x$ and $\tt u$ are terms, then $\tt (t\ u)$ is a term.
    \item If $\tt x$ is a variable and $\tt t,T,u$ are terms, then $\tt let\ \ x:=t:T\ \ in\ \ u$ is a term.
\end{enumerate}

\subsection{Typing Rule}

\subsubsection{Notation}
\begin{itemize}
\item $\Gamma$ : local context.
\item $\tt u\{x/t\}$ : substitute free occurrence of variable $\tt x$ to term $\tt t$ in term $\tt u$.
\item $\mathcal{WF}(\Gamma)$ : $\Gamma$ is well-formed.
\end{itemize}

\subsubsection{Typing Rules}
\begin{equation*}
\mathcal{WF}([])[] 
    \tag{\sc T-Empty}
\end{equation*}
\begin{equation*}
\frac{\Gamma\vdash {\tt T:Type\qquad \qquad x}\notin\Gamma}
    {\mathcal{WF}(\Gamma ::(\tt x:T))} 
    \tag{\sc T-Ax}
\end{equation*}
\begin{equation*}
\frac{\Gamma\vdash {\tt t:T\qquad c}\notin \Gamma}
    {\mathcal{WF}(\Gamma:\tt c:=t:T)} 
    \tag{\sc T-Def}
\end{equation*}
\begin{equation*}
\frac{\mathcal{WF}(\Gamma)\qquad ({\tt x:T})\in\Gamma}
    {\Gamma\vdash\tt x:T} 
    \tag{\sc T-Var1}
\end{equation*}
\begin{equation*}
\frac{\mathcal{WF}(\Gamma)\qquad ({\tt x:=t:T})\in\Gamma}
    {\Gamma\vdash\tt x:T} 
    \tag{\sc T-Var2}
\end{equation*}
\begin{equation*}
\frac{\mathcal{WF}(\Gamma)\qquad ({\tt c:T})\in \Gamma}
    {\Gamma\vdash\tt c:T} 
    \tag{\sc T-Const1}
\end{equation*}
\begin{equation*}
\frac{\mathcal{WF}(\Gamma)\qquad ({\tt c:=t:T})\in \Gamma}
    {\Gamma\vdash\tt c:T} 
    \tag{\sc T-Const2}
\end{equation*}

\begin{equation*}
\frac{\Gamma\vdash{\tt T:Type}\qquad \Gamma::({\tt x:T})\vdash\tt U:Type}
    {\Gamma\vdash\tt\forall x:T,U:Type} 
    \tag{\sc T-Prod}
\end{equation*}
\begin{equation*}
\frac{\Gamma\vdash\forall{\tt x:T,U:Type}\qquad \Gamma::({\tt x:T})\vdash{\tt t:U}}
    {\Gamma\vdash\lambda{\tt x:T.\ t:\forall x:T,U}} 
    \tag{\sc T-Abs}
\end{equation*}
\begin{equation*}
\frac{\Gamma\vdash\forall{\tt x:U,T}\qquad \Gamma\vdash{\tt u:U}}
    {\Gamma\vdash{\tt (t\ u):T\{x/u\}}} 
    \tag{\sc T-App}
\end{equation*}
\begin{equation*}
\frac{\Gamma\vdash{\tt t:T}\qquad \Gamma::({\tt x:=t:T})\vdash{\tt u:U}}
    {\Gamma\vdash{\tt let\ \ x:=t:T\ \ in\ \ u:U\{x/t\}}} 
    \tag{\sc T-Let}
\end{equation*}
\begin{equation*}
\frac{(\Gamma\vdash {\tt A_i:s_i)_{i=1..n}}\qquad (\Gamma,\tt f_1:A_1,\cdots,f_n:A_n\vdash t_i:A_i)_{i=1..n}}
    {\Gamma\vdash\tt Fix\ f_i\{f_1:A_1:=t_1\cdots f_n:A_n:=t_n\}:A_i}
    \tag{\sc T-Fix}
\end{equation*}

\subsection{Inductive Definition}

\subsubsection{Notation}
\begin{itemize}
    \item ${\tt Ind}[p](\Gamma_I:=\Gamma_C)$ : inductive definition.
    \item $\Gamma_I$ : names and types of inductive type.
    \item $\Gamma_C$ : names and types of constructors of inductive type.
    \item $p$ : the number of parameters of inductive type.
    \item $\Gamma_P$ : the context of parameters.
\end{itemize}

\subsubsection*{Example}
Here is an example:
\begin{center}
\begin{minipage}{0.6\textwidth}
\begin{minted}{coq}
Inductive list (T : Type) : Type :=
| nil : list T
| cons : T -> list T -> list T.
\end{minted}
\end{minipage}
\end{center}
$$
\tt Ind\ [1]\left([list:Type\to Type]:=
\begin{bmatrix}
\tt nil:\forall T:Type,\ list\ T\\
\tt cons:\forall T:Type,\ T\to list\ T\to list\ T
\end{bmatrix}\right)
$$
Sadly, due to our limited time and energy, our system only supports inductive definition with $|\Gamma_I|=1$,
which means inhibiting mutual inductive type like the following case:
\begin{center}
\begin{minipage}{0.6\textwidth}
\begin{minted}{coq}
Inductive tree : Type :=
    node : forest -> tree
with forest : Type :=
    | emptyf : forest
    | consf : tree -> forest -> forest.
\end{minted}
\end{minipage}
\end{center}
$$
\tt Ind\ []\left(
\begin{bmatrix}\tt tree:Type\\\tt forest:Type\end{bmatrix}
:=
\begin{bmatrix}
\tt node:forest\to tree\\
\tt emptyf:forest\\
\tt consf:tree\to forest\to forest
\end{bmatrix}\right)
$$

\subsubsection{Typing Rule}
\begin{equation*}
\frac{\mathcal{WF}(\Gamma)\qquad {\tt Ind}[p](\Gamma_I:=\Gamma_C)\in \Gamma\qquad {\tt(a:A)}\in\Gamma_i}
    {\Gamma\vdash\tt a:A}
    \tag{\sc T-Ind}
\end{equation*}
\begin{equation*}
\frac{\mathcal{WF}(\Gamma)\qquad {\tt Ind}[p](\Gamma_I:=\Gamma_C)\in \Gamma\qquad {\tt (c:C)}\in\Gamma_C}
    {\Gamma\vdash\tt c:C}
    \tag{\sc T-Constr}
\end{equation*}
\begin{equation*}
\frac{(\Gamma_P\vdash{\tt A_j:s_j'})_{j=1..k}\qquad (\Gamma_i;\Gamma_P\vdash{\tt C_i:s_{q_i}})_{i=1..n}}
    {\mathcal{WF}(\Gamma;{\tt Ind}[p](\Gamma_I:=\Gamma_C))}
    \tag{\sc T-Wf-Ind}
\end{equation*}


\subsection{Match}

\subsubsection{Notation}
The basic idea of this operator is that we have an object $\tt m$ in an inductive type $\tt I$ and we want
to prove a property which possibly depends on $\tt m$. For this, it is enough to prove the property for
$\tt m=(c_i\ u_1\ \cdots\ u_{p_i})$ for each constructor of $\tt I$.
The term for this proof will be written:
$$
\tt match\ m\ in\ I\ \_\ a\ return\ P\ with\ (c_1\ x_{11}\ \cdots\ x_{1p_1})\Rightarrow f_1\ |\ \cdots\ |\ (c_n\ x_{n1}\ \cdots\ x_{np_n})\Rightarrow
f_n\ end
$$\par
Note that the arguments of $\tt I$ corresponding to parameters must be $\tt\_$, because the result type is not 
generalized to all possible values of the parameters. The other arguments of $\tt I$ (sometimes called \textit{indices}
in the literature) have to be variables ($\tt a$ above) and these variables can occur in $\tt P$. The expression after 
$\tt in$ must be seen as an inductive type pattern. Notice that expansion of implicit arguments and notations apply 
to this pattern. For the purpose of presenting the inference rules, we use a more compact notation:
$$
\tt case(m,(\forall ax.\ P),\lambda x_{11}\cdots x_{1p_1}.\ f_1\ |\ \cdots\ |\ \lambda x_{n1}\cdots x_{np_n}.\ f_n)
$$

\subsubsection{Type of Branch}
Before coming to the typing rule of {\tt match}, we have to deal with the type of every branch case of {\tt match}.\par
Let $\tt c$ be a term of type $\tt C$, assume $\tt C$ is a type of constructor for an inductive type $\tt I$. Let
$\tt P$ be a term that represents the property to be proved. Assume $\tt r$ is the number of parameters and 
$p$ is the number of arguments.\par
Define a new type $\tt \{c:C\}^P$ which represents the type of the branch corresponding to the $\tt c:C$ constructor.
\begin{center}
\begin{tabular}{lcl}
$\tt \{c:(I\ p_1\ \cdots\ p_r\ t_1\ \cdots\ t_p)\}^P$ & $\equiv$ & $\tt (P\ t_1\ \cdots\ t_p\ c)$\\
$\tt \{c:\forall x:T,\ C\}^P$ & $\equiv$ & $\tt \forall x:T, \{(c\ x):C\}^P$
\end{tabular}
\end{center}
We write $\tt\{c\}^P$ for $\tt\{c:C\}^P$ with $\tt C$, the type of $\tt c$.

\subsubsection{Typing Rule}
\begin{equation*}
\frac{\Gamma\vdash{\tt c:(I\ q_1\ \cdots\ q_r\ t_1\ \cdots\ t_s)}\qquad 
        \Gamma\vdash{\tt [P\ |\ I\ q_1\ \cdots\ q_r]}\qquad
    (\Gamma\vdash{\tt f_i:\{(c_{p_i}\ q_1\ \cdots\ q_r)\}^P})_{i=1..l}}
    {\Gamma\vdash{\tt case(c,P,f_1\ |\ \cdots\ |\ f_l):(P\ t_1\ \cdots\ t_s\ c)}}
    \tag{\sc T-Match}
\end{equation*}
$\tt [A\ |\ B]$ means the consequent of $\tt A$ is $\tt B$.
