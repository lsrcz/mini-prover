\section{Top Level}

In top level, the system maintains the interaction with user and
adds the user input, namely, \textit{Axiom}, \textit{Definition}, \textit{Fixpoint}, and \textit{Inductive Definition}, 
into the context.\par
It is also in this part, the inductive rule forms.

\subsection{Axiom}
After parsing and checking, the command will be like
$$
\tt Ax\ \ {\it name\ term},
$$ 
where {\it term} denotes the type of this axiom.
Since it is an axiom, we do not have to (sometimes can not) build the corresponding term.\par
Just build the corresponding term as {\it Nothing} then put it into the context.

\subsection{Definition}
After parsing and checking, the command will be like
$$
\tt Def\ \ {\it name\ term2\ term1},
$$
where {\it term2} is the type of {\it term1}.\par
Simply bind the name, term, type together and put it into the context.

\subsection{Fixpoint}
After parsing and checking, the command will be like
$$
\tt Fix\ {\it name }\ \left(\lambda {\it f}:{ \it term1},\ {\it term2}\right),
$$
where {\it term1} is the type of {\it term2} and it is a recursive function of {\it f}.\par 
Since this recursive function has passed all the type check and safety check, we can safely use it without worrying termination
problem.
On the other hand, whether it is a {\it Fixpoint} definition will not influence any reduction, because the reduction
always finds the term in context according to its index.\par
So simply remove the {\it Fixpoint} mark and put it into the context.

\subsection{Inductive Definition}
After parsing and checking, the command will be like
$$
\tt Ind\ \ {\it name\ p\ term2\ term1\ constructors},
$$
where {\it p} is the number of parameters of the inductive type, {\it term1} is the type of the inductive definition,
and {\it term2} is the corresponding term.\par
Apart from the ordinary operation, we also need to add inductive rule, which is actually a type theory view of mathematical
induction. Since the proof of a claim becomes a term of certain type, the induction rule is a term offering inductive scheme.\par
For example,
\begin{center}
\begin{minted}{coq}
Inductive nat : Type := 
| O : nat
| S : nat -> nat

(* build inductive rule*)

fun (P:nat -> Type)(f:P O)(f0:forall (n:nat), P n -> P (S n))
    fix F (n:nat) : P n :=
        match n as n0 in nat return (P n0) with
        | O => f
        | S n0 => f0 n0 (F n0)
        end
:
forall (P:nat -> Type), P O -> 
    (forall (n:nat), P n -> P (S n)) ->
        forall (n:nat), P n
\end{minted}
\end{center}
The intuition here is that for a proposition \mintinline{coq}|P|,
\begin{itemize}
\item it is true on \mintinline{coq}|O|;
\item if it is true on \mintinline{coq}|n|, then it is true on \mintinline{coq}|S n|.
\end{itemize}
Then it is true on all term of \mintinline{coq}|nat| type, which is reasonable according 
mathematical induction.\par
Basically, to build such term, we should build weakened assumptions for all constructors first, 
like the \mintinline{coq}|f|,
\mintinline{coq}|f0| above. After that, the final proposition which applies to all the terms of such inductive type
shall come out, like the \mintinline{coq}|F| above.\par
Every occurrence of the inductive type on the constructors demands a verification of the proposition,
which explains why for constructor \mintinline{coq}|S : nat -> nat|, which depends on a \mintinline{coq}|nat| term, 
the weakened assumption is \mintinline{coq}|f0 : forall (n:nat), P n -> P (S n)|.\par
The reason why the inductive rule of \mintinline{coq}|nat| requires {\it Fixpoint} is that some constructors of it
rely on the term of type \mintinline{coq}|nat|. Here is a case which do not need recursive function.
\begin{center}
\begin{minted}{coq}
Inductive eq (T : Type) (x : T) : T -> Type :=
  | eq_refl : eq T x x

(* build inductive rule*)

fun (T:Type) (x:T) (P:forall (t:T) (_:eq T x t), Type) 
    (f:P x (eq_refl T x)) (t:T) (e:eq T x t) => 
        match e as e0 in eq _ _ a0 return (P a0 e0) with 
        | eq_refl _ _ => f
        end
: 
forall (T:Type) (x:T) (P:forall (t:T) (_:eq T x t), Type) 
    (f:P x (eq_refl T x)) (t:T) (e:eq T x t),
        P t e
\end{minted}
\end{center}
Sadly, we have to admit that because of the lack of references, time, and energy,
the induction rule in our system is not complete.
The inductive definition acceptable to our system must satisfy:\par
Assume the inductive type is $\tt A_1\to A_2\to \cdots A_n$, then $\tt A_k$ must be
any one of the following
\begin{itemize}
\item A ordinary term, like \mintinline{coq}|Type|, \mintinline{coq}|T|.
\item An application, like \mintinline{coq}|P n|.
\item An inductive type, like \mintinline{coq}|nat|, \mintinline{coq}|eq T x y|.
\end{itemize}
Others like product type \mintinline{coq}|U -> V| is not supported.


\subsection{Top Loop}
