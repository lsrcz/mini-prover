\section{Safety Check}

Naturally, we prefer error alerts right after inputting rather than let the system run till crash. 
Parser helps filter bad inputs, type check verifies the transaction between term and term, while 
neither of them guarantees termination.\par
In this part, we will demonstrate how to prevent non-halting definition from getting accepted into
the system.

\subsection{Fixpoint}

The introduction of {\it Fixpoint} gives a wide range of flexibility of the program, without which recursive
function can not formulate. On the other hand, the risk of non-terminating appears.\par
To avoid this, we put a strong constraint on recursive function, which is that it must descend on at least one
argument. 
\begin{Prop}[Termination of Fixpoint]
If a recursive function descends on at least one argument, it will always terminate in reduction.
\end{Prop}
\begin{Def}[Descending]
Recursive function $f(x_1,x_2,\cdots,x_n)$ is descending on $x_k$ if any occurrence of $f$ in function definition has the 
form $f(y_1,y_2,\cdots,y_n)$, where $y_k$ is inferior to $x_k$.
\end{Def}
\begin{Def}[Inferior]
In a function definition, $x$ is inferior to $y$ if $y\to_m y_1\to_m \cdots\to_m y_s\equiv x$.
$a \to_m b$ means a pattern match directly from $a$ to $b$.
\end{Def}
Here is an example.
\begin{center}
\begin{minted}{coq}
(* accepted *)                                 (* rejected *)
Fixpoint plus (n:nat)(m:nat) : nat :=          Fixpoint plus (n:nat)(m:nat) : nat :=
    match n as n0 in nat return (nat) with         match n as n0 in nat return (nat) with    
    | O => m                                       | O => m
    | S p => S (plus p m)                          | S p => S (plus m p)
    end.                                           end.
\end{minted}
\end{center}

\subsubsection*{Implementation}

Check every argument of the recursive function. For a specific argument, maintain a list of terms inferior to it 
during the process and verify every occurrence of the recursive function.

\subsection{Inductive Type}

The inductive type is actually a recursive definition, which is so powerful that we can create non-terminating program
without actually defining a recursive function.
\begin{center}
\begin{minipage}{0.7\textwidth}
\begin{minted}{coq}
Inductive ill : Type :=                
| malf : (ill -> ill) -> ill.          
                                       
Definition extract (t:ill) : ill :=       
    match t as t0 in ill return (ill) with 
    | malf f => f t
    end.                                    

extract (malf extract)  (* not terminating *)
\end{minted}
\end{minipage}
\end{center}

\begin{Prop}[Termination of Inductive Type]
If the type of constructor of the inductive definition satisfies the positivity condition for
the inductive type, it will always terminate in reduction.
\end{Prop}

\begin{Def}[Positivity]
The type of constructor $\tt T$ satisfies \textit{the positivity condition} for a constant $\tt X$ if
\begin{itemize}\normalfont
    \item $\tt T\equiv (X\ t_1\ \cdots\ t_n)$ and $\tt X$ does not occur free in $\tt t_i$
    \item $\tt T\equiv \forall x:U,V$ and $\tt X$ occurs only \textit{strictly positively} in $\tt U$ and
        $\tt V$ satisfies \textit{the positivity condition} for $\tt X$
\end{itemize}
\end{Def}

\begin{Def}[Strictly Positivity]
The constant $\tt X$ occurs \textit{strictly positively} in $\tt T$ if
\begin{itemize}\normalfont
    \item $\tt X$ does not occur in $\tt T$
    \item $\tt T\mathop{\rhd}^*\tt (X\ t_1\ \cdots\ t_n)$ and $\tt X$ does not occur in $\tt t_i$
    \item $\tt T\mathop{\rhd}^*\tt \forall x:U,V$ and $\tt X$ does not occur in $\tt U$ but occurs
        \textit{strictly positively} in $\tt V$
    \item $\tt T\mathop{\rhd}^*\tt (I\ a_1\ \cdots\ a_m\ t_1\ \cdots\ t_p)$, where
        ${\tt Ind}[m](\tt I:A:=c_1:\forall p_1:P_1,\dots\forall p_m:P_m,C_1;\cdots;c_n:\forall p_1:P_1,\dots,
        \forall p_m:P_m,C_n)$, and $\tt X$ does not occur in $t_i$, and the types of constructor
        $\tt C_i\{p_j/a_j\}_{j=1..m}$ satisfies \textit{the nested positivity condition} for $\tt X$
\end{itemize}
\end{Def}

\begin{Def}[Nested Positivity]
The type of constructor $\tt T$ satisfies \textit{the nested positivity condition} for a constant $\tt X$ if
\begin{itemize}\normalfont
    \item $\tt T\equiv(I\ b_1\ \cdots\ b_m\ u_1\ \cdots\ u_p)$, where $\tt I$ is an inductive definition with $m$
        parameters and $\tt X$ does not occur in $u_i$
    \item $\tt T\equiv\forall x:U,V$ and $\tt X$ occurs \textit{strictly positively} in $\tt U$ and $\tt V$
        satisfies \textit{the nested positivity condition} for $\tt X$
\end{itemize}
\end{Def}
